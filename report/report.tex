\documentclass[10pt, a4paper,openany]{article}
\usepackage[italian]{babel}
\usepackage[T1]{fontenc}
\usepackage[table]{xcolor}
\usepackage{float}
\restylefloat{table,figure}
\usepackage{graphicx}	
\usepackage[utf8]{inputenc}
\usepackage{amsmath}
\usepackage{fancyhdr}
\usepackage{geometry}
\usepackage{url}
\usepackage{hyperref}
\usepackage[ruled,vlined]{algorithm2e}
\geometry{a4paper,top=2cm,bottom=2cm,left=3cm,right=3cm,%
	heightrounded,bindingoffset=5mm}
\usepackage{amssymb}
\usepackage{amsthm}
\usepackage{blindtext}

\usepackage{multicol}

\begin{document}

\begin{center}
\huge{\textbf{Progetto AML}}

SOTTOTITOLO
\end{center}

\begin{center}
Federico Luzzi 816753, Christian Uccheddu 800428
\end{center}

\hrule
\vspace{0.2cm}
\begin{center}\textbf{Introduzione e obiettivi}\end{center} 
\blindtext[1]
\\\\ \begin{small}
	\textit{Keyword: Covid-19, YouTube}
\end{small}
\vspace{0.2cm}
\hrule
\vspace{0.2cm}

\begin{multicols}{2}
	\section{uiassahn}
	\blindtext[3]
	\begin{figure}[H]

		\centering

		\includegraphics[height=0.6 \linewidth]{pics/IterHold.png}

		\caption{Il grafico mostra la variazione tra le accuracy di train (sinistra) e test (destra) per ogni modello}

		\label{fig:trainHold}

	\end{figure}
	aisnsadniasd \cite{a:tesi}
\end{multicols}

\newpage
\begin{thebibliography}{1}

	\bibitem{b:tesi}\href{https://it.wikipedia.org/wiki/Demenza}{\emph{Demenza - \url{https://it.wikipedia.org/wiki/Demenza}}}
	
\end{thebibliography}

\end{document}